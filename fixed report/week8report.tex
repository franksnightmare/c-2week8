\documentclass[11pt]{article}

\usepackage{times}
\usepackage[english]{babel}

% -----------------------------------------------
% especially use this for you code
% -----------------------------------------------

\usepackage{courier}
\usepackage{listings}
\usepackage{color}
\usepackage{tabularx}
\usepackage{graphicx}

\definecolor{Gray}{gray}{0.95}

\definecolor{mygreen}{rgb}{0,0.6,0}
\definecolor{mygray}{rgb}{0.5,0.5,0.5}
\definecolor{mymauve}{rgb}{0.58,0,0.82}

\lstset{language=C++,
	basicstyle = \normalsize\ttfamily,   % the size and fonts that are used
	tabsize = 2,                    % sets default tabsize
	breaklines = true,              % sets automatic line breaking
	keywordstyle=\color{blue}\ttfamily,
	stringstyle=\color{red}\ttfamily,
	commentstyle=\color{mygreen}\ttfamily,
	numbers=left,
	keepspaces=true,
	showspaces=false,
	showstringspaces=false,
}

\begin{document}

\title{Programming in C/C++ \\
       Exercises set eight: function templates
}
\date{\today}
\author{Christiaan Steenkist \\
Jaime Betancor Valado \\
Remco Bos \\
}

\maketitle
\section*{Exercise 56, cast as}
Let's be honest, static casts are a pain.
We tried to alleviate this pain by shortening it to a simple as.

\subsection*{Code listings}
\lstinputlisting[caption = main.ih]{src/a56/main.ih}
\lstinputlisting[caption = main.h]{src/a56/main.h}
\lstinputlisting[caption = main.cc]{src/a56/main.cc}

\section*{Exercise 57, code bloat away}
Gee who would have thunked that the compiler is actually smart and combines two identical instances of a template function together into a single functino saving code space and preventing code bloat? I SURE would.

\subsection*{Output}
\begin{lstlisting}
0x804873e
0x804873e
\end{lstlisting}

\subsection*{Code listings}
\lstinputlisting[caption = main.h]{src/a57/main.h}
\lstinputlisting[caption = pointerunion.h]{src/a57/pointerunion.h}
\lstinputlisting[caption = addtwo.h]{src/a57/addtwo.h}
\lstinputlisting[caption = add1.cc]{src/a57/add1.cc}
\lstinputlisting[caption = add2.cc]{src/a57/add2.cc}
\lstinputlisting[caption = main.cc]{src/a57/main.cc}
\section*{Exercise 58, function templates}
We were tasked to modify the definition of operator overloading in exercise 36 using function templates.
\subsection*{Code listings}
\lstinputlisting[caption = exception.h]{src/a58/exception.h}
\subsection*{Exercuse 60, GUARDS!}
Whether you are using a mutex or a smaphore, this guard's got your back.

\lstinputlisting[caption = main.ih]{src/a60/main.ih}
\lstinputlisting[caption = main.h]{src/a60/main.h}
\lstinputlisting[caption = main.cc]{src/a60/main.cc}

\subsubsection*{Semaphore}
\lstinputlisting[caption = main.ih]{src/a60/semaphore.ih}
\lstinputlisting[caption = main.h]{src/a60/semaphore.h}
\lstinputlisting[caption = mutexref.cc]{src/a60/mutexref.cc}

\subsubsection*{Guard}
\lstinputlisting[caption = guard.ih]{src/a60/guard.ih}
\lstinputlisting[caption = guard.h]{src/a60/guard.h}
\lstinputlisting[caption = constructor1.cc]{src/a60/guardconstr1.cc}
\lstinputlisting[caption = constructor2.cc]{src/a60/guardconstr2.cc}
\lstinputlisting[caption = paroperator.cc]{src/a60/paroperator.cc}

\section*{Exercise 62, function selection}
Here some questions and some answers, thats it.
\subsubsection*{Question 1}
Q: Why is the scope resolution operator required when calling max()?
A: Because max is defined in the global namespace.

\subsection*{Question 2}
Q: When compiling this function the compiler complains with a message like:
\begin{lstlisting}
max.cc:13: error: no matching function for call to 'max(double, int)'
\end{lstlisting}
Why doesn't the compiler generate a max(double, double) function in this case? 

A: The compiler first sees a double and decides that the typename Type must be double.
The compiler will not promote the int to a double by itself, because it checks if the int is a double, which it isn't.

\subsection*{Question 3}
Q: Assume we add a function
\begin{lstlisting}
double max(double const &left, double const &right)
\end{lstlisting}
to the source. Explain why this solves the problem.

A: The compiler does not have to check if both values are the same type (i.e. double in the previous question), so it can convert the int into a double. 

\subsection*{Question 4}
Q: Assume we would then call \texttt{::max('a', 12)}. Which \texttt{max()} function is then used and why?
A: It will use the first one, the template. Because 'a' is a char and not a double.

\subsection*{Question 5}
Q: Remove the additional max function. Without using casts or otherwise changing the argument list of the function call \texttt{max(3.5, 4)}, how can we get the compiler to compile the source properly?
A: Use this line: \texttt{cout << ::max<double>(3.5, 4) << endl;}

\lstinputlisting[caption = main.cc]{src/a62/main.cc}

\subsection*{Question 6}
Q: Specify a general characteristic of the answer to the previous question (i.e., can the approach always be used or are there certain limitations?).
A: The characteristic is that the template is only used withthe specified type(in this case double). If some variable can be converted to the specified type, then it will always work. A limitation is when the compiler cannot convert a type to another type(e.g. a str::string to int).

\end{document}